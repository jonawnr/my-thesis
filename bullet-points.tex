\documentclass[sigplan, nonacm]{acmart}

\begin{document}

\title{Model-based Mutation Testing: A Comparative Study with PITest}

\author{Jonathan Wagner}
\email{wagnerjo@students.uni-marburg.de}
\affiliation{%
	\institution{Philipps-University Marburg}
	\city{Marburg}
	\country{Germany}
}

\maketitle

\section{Introduction}
\begin{itemize}
	\item Definition and overview of Mutation Testing
	\item Importance of Mutation Testing in software testing
	\item Brief motivation for comparing MMT and PITest
\end{itemize}

\section{Mutation Testing}
\begin{itemize}
	\item Basic concepts (mutants, mutation operators, mutation score)
	      \begin{itemize}
		      \item What are mutants?
		      \item What are mutation operators and examples?
		      \item What is the mutation score and why is it not always the best metric?
	      \end{itemize}
	\item Advantages and limitations
	      \begin{itemize}
		      \item Advantages:
		      \item High Quality Tests
		      \item Detects Weaknesses missed by Traditional Coverage Metrics
		      \item Improves Code Quality
		      \item Objective Assessment
		      \item Effective in Critical Systems
		      \item Disadvantages:
		      \item High Computational Cost, especially for MMT and for large codebases
		      \item Equivalent Mutants Problem
		      \item Requires Robust Infrastructure. Setting up can be complex and maintanance-intensive. May require significant investment in tooling and integration.
		      \item Interpretation Challenges. 100\% Mutation Score not always feasible or desirable
		      \item Overly Complex Tests can slow down development and decrease maintainability.
	      \end{itemize}
\end{itemize}

\section{Model-based Mutation Testing Tool (MMT)}
\begin{itemize}
	\item Definition and overview of MMT
	\item Key features and functionalities of MMT
	\item Current status and usage context (research or industrial)
	\item Recent improvements or ongoing developments
\end{itemize}

\section{Analyzing the MMT Model}
\begin{itemize}
	\item Description of the underlying model (graph, UML, FSM, etc.)
	\item Visual representation of a typical MMT model (example figure later)
	\item Metrics applied to MMT model (complexity metrics, coverage metrics, etc.)
	\item Analysis of model metrics indicating performance limitations
	\item Suggestions for performance improvement based on metric analysis
\end{itemize}

\section{PITest}
\begin{itemize}
	\item Definition and overview of PITest
	\item Key capabilities and functionality
	\item Current status and community adoption (industrial/research)
	\item Notable recent improvements and community-driven extensions
\end{itemize}

\section{Applying Mutation Testing to Defects4J}
\begin{itemize}
	\item Introduction to Defects4J (brief)
	\item Methodology for applying MMT and PITest to Defects4J examples
	\item Summary of expected outcomes or preliminary findings
\end{itemize}

\section{Comparison of MMT with PITest}
\begin{itemize}
	\item Feature-by-feature comparison (mutation operators, ease-of-use, scalability)
	\item Performance comparison (speed, mutation score, mutant generation efficiency)
	\item Strengths and weaknesses of each tool
	\item Contextual suitability (when to prefer MMT or PITest)
\end{itemize}

\section{Conclusion}
\begin{itemize}
	\item Summary of main points discussed
	\item Highlight potential improvements for MMT identified by metrics
	\item Recommendations for future research and practical application
\end{itemize}

\bibliographystyle{ACM-Reference-Format}
\bibliography{references}

\end{document}
